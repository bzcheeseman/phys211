\documentclass{beamer}
\usepackage{geometry}
\usepackage{graphicx, dblfloatfix}
\usepackage{amsmath, amssymb, amsfonts}

\title{Gamma Cross Sections}
\author{Aman LaChapelle}
\institute{University of Chicago}
\date{October 21, 2015}

%\AtBeginSection[]
%{
%  \begin{frame}
%    \frametitle{Table of Contents}
%    \tableofcontents[currentsection]
%  \end{frame}
%}
\AtBeginSubsection[]
{
	\begin{frame}
		\frametitle{Table of Contents}
		\tableofcontents[currentsection, currentsubsection]
	\end{frame}
}

\DeclareMathOperator\cov{cov}

\begin{document}

\frame{
	\titlepage
}

\section{PHA spectra}
	\frame{
		\frametitle{Gaussian Fits}
		\begin{equation*}
			f(x) = \frac{N}{\sigma\sqrt{2\pi}} e^{ \frac{-(x-\mu)^2}{2\sigma^2} } + Ax + B
		\end{equation*}

		\vspace{1cm}

		We fit all the peaks to this function because we didn't trust the ROI settings and their consistency.
	}

	\frame{
		\frametitle{Gaussian Fits}
		\begin{figure}[!htb]
			\centering
			\includegraphics[height=7.5cm]{../plots/Ba40mmAl.pdf}
		\end{figure}

		An example fit.
	}

	\subsection{Distribution Width}
	\frame{
		\frametitle{Distribution Width}
		\begin{figure}[!htb]
			\centering
			\includegraphics[height=4cm]{../plots/Ba40mmAl.pdf}
		\end{figure}

		The distribution should be a delta function - it isn't because of the detector resolution
	}

	\frame{
		\frametitle{Distribution Width}

		\begin{center}
		\begin{tabular}{|l|l|l|}
			\hline
			Energy (keV) & $\Gamma$ (channels) & $\Gamma/E$\\ \hline \hline
			31 & 8.36 & 26.9\%\\
			81 & 10.18 & 12.6\%\\
			356 & 30.25 & 8.49\%\\ \hline
			511 & 10.68 & 2.09\%\\
			1270 & 16.69 & 1.31\%\\
			\hline
		\end{tabular}
		\end{center}

		Not perfect resolution - this is our $\sigma$ (up to $\Gamma/\sqrt{2}\sqrt{ln(2)}$).
	}

	\subsection{Error Propagation}

	\frame{
		\frametitle{Covariance Matrix}
		Scipy and Numpy return the covariance matrix, but here is some information on what it is:
		\begin{gather*}
			\Sigma_{ij} = (X_i - \mu_i)(X_j - \mu_j)\\
			\Sigma_{ij} \equiv \cov(i,j)
		\end{gather*}
		with each $\mu_k$ being what we expect that datum to be assuming it fits perfectly to our model.  We can express each X as a vector and have a matrix equation or we can build it up piece-by-piece.  In general least-squares optimization we use
		\begin{gather*}
			C(x^*) = (J^T(x)J(x))^{-1}
		\end{gather*}
		where $J(x)$ is the jacobian of the model function at the point x.  $x^*$ is the maximum likelihood estimate, $min ||f(x)||^2$. \cite{ceres}
	}

	\frame{
		\frametitle{Error Propagation of the PHA fits}
		Since we performed fits, the error of the ith parameter is just:
		\begin{equation*}
			\sigma_{i} = \sqrt{cov(i,i)}
		\end{equation*}

		Because our largest $\delta t$ was less than 0.001\% the uncertainty in parameter i is just $\sigma_{i}$
	}

	\frame{
		\frametitle{Error Propagation of the Countrate Fits}

		We use the same technique as we did with the PHA spectra to calculate the uncertainty in our parameters.  A sample of the data used to calculate the countrates:

		\begin{tiny}
		\begin{center}
		Ba-133 Countrates and Uncertainty\\
		\begin{tabular}{|l|l|l|l|l|l|l|}
			\hline
			Thickness (mm) & N1 (count/s) & N2 (count/s) & N3 (count/s) & $\delta$N1 & $\delta$N2 & $\delta$N3\\ \hline \hline
			56 & 9.044 & 9.910 & 39.361 & 1.915\% & 4.832\% & 1.959\%\\ \hline
			40 & 13.480 & 20.943 & 57.620 & 2.632\% & 3.389\% & 2.156\%\\ \hline
			... & ... & ... & ... & ... & ... & ...\\ \hline
			2 & 303.314 & 145.682 & 156.056 & 1.094\% & 1.302\% & 2.127\%\\ \hline
			1 & 391.492 & 152.985 & 155.885 & 1.032\% & 1.320\% & 2.074\%\\
			\hline
		\end{tabular}
		\end{center}
		\end{tiny}

		Our error bars in the countrate data are exactly the percent error you see above, and we will see this moving forward.
	}


\section{Countrates}
	\subsection{Na}
	\frame{
		\frametitle{Fitting to find $\lambda$}

		We fit to the function 
		\begin{equation*}
			R(x) = R_0e^{-\lambda x} + C
		\end{equation*}
		to extract our values for $\lambda$
	}

	\frame{
		\frametitle{Fitting to find $\lambda$}
		\begin{figure}[!htb]
			\centering
			\includegraphics[height=6cm]{../plots/countrates_na.pdf}
		\end{figure}

		\begin{center}
		\begin{tabular}{|l|l|l|}
			\hline
			Energy & $\lambda$ & $\delta \lambda$\\ \hline
			511 keV & $0.19 \, cm^{-1}$ & $0.03 \, cm^{-1}$\\
			1.27 MeV & $0.09  \, cm^{-1}$ & $0.05 \, cm^{-1}$\\
			\hline
		\end{tabular}
		\end{center}
	}

	\subsection{Ba}
	\frame{
		\frametitle{Linear Attenuation Coefficient Character}
		\begin{figure}
			\centering
			\includegraphics[height=7cm]{../report/linear_attenuation_coefficient_al.pdf}
		\end{figure}
	}
	\frame{
		\frametitle{Fitting to find $\lambda$}

		We fit to the function
		\begin{equation*}
			R(x) = R_0e^{-\lambda x} + R_0'e^{-\tau x} + C
		\end{equation*}
		with $\lambda$ being the compton-like attenuation and $\tau$ being the photoelectric-like attenuation.
	}

	\frame{
		\frametitle{Fitting to find $\lambda$}

		\begin{figure}[!htb]
			\centering
			\includegraphics[height=6cm]{../plots/countrates_ba.pdf}
		\end{figure}

		\begin{center}
		\begin{tabular}{|l|l|l|}
			\hline
			Energy & $\lambda$ & $\delta \lambda$\\ \hline
			31 keV & $2.9 \, cm^{-1}$ & $0.1 \, cm^{-1}$\\
			81 keV & $0.513  \, cm^{-1}$ & $0.06 \, cm^{-1}$\\
			356 keV & $0.24 \, cm^{-1}$ & $0.04 \, cm^{-1}$\\
			\hline
		\end{tabular}
		\end{center}
	}
	\subsection{Comparison to Literature}
	\frame{
		\frametitle{Comparison to Literature}

		\begin{figure}[!htb]
			\centering
			\includegraphics[height=8cm]{../plots/confidence.pdf}
		\end{figure}
	}

	\frame{
		\frametitle{Comparison to Literature}
		\begin{small}
		\begin{center}
		\begin{tabular}{|l|l|l|l|}
			\hline
			Energy & $\lambda \pm \delta \lambda  \, (cm^{-1})$ & $\lambda_{calc} \pm \delta\lambda_{calc}  (\, cm^{-1})$ & Confidence Level\\ \hline \hline
			31 keV & $2.9 \pm 0.1$ & $2.98 \pm 0.03$ & 57.6\%\\
			81 keV & $0.513 \pm 0.06$ & $0.533 \pm 0.03$ & 66.8\%\\
			356 keV & $0.24 \pm 0.04$ & $0.3 \pm 0.1$ & 79.0\%\\
			511 keV & $0.19 \pm 0.03$ & $0.22 \pm 0.05$ & 42.1\%\\
			1.27 MeV & $0.09 \pm 0.05$ & $0.15 \pm 0.07$ & 33.8\%\\
			\hline
		\end{tabular}
		\end{center}
		\end{small}

		The NIST values have uncertainties calculated based on how far the tabulated energy was from our energy and estimating the slope of the area immediately around the point of interest.
	}
	\subsection{Error Propagation/Calculation}
	\frame{
		\frametitle{Systematic Uncertainty}
		We noticed that our data (for $\lambda$) was consistently lower than the NIST values, so we tried to fit it and the NIST values to:*
		\begin{equation*}
			f(x) = Ae^{Bx} + Ce^{Dx} + E.
		\end{equation*}
		We found that the difference between our values and the NIST values was in E - it was just an offset.  Furthermore we got a value for E of $0.020\pm0.003$ and so we added that 0.02 to all our uncertainties throughout for $\lambda$.

		\vspace{1cm}

		\tiny{* not shown}
	}
	

\section{References}
	\frame{
		\frametitle{References}
		\begin{thebibliography}{10}
			\setbeamertemplate{bibliography item}[website]
			\bibitem{physics.nist.gov}
				\small{Physics.NIST.gov - Table of XRay Mass Attenuation Coefficients\\
				http://physics.nist.gov/PhysRefData/XrayMassCoef/ElemTab/z13.html}
			\bibitem{lab manual}
				\small{University of Chicago, PHYS 211 Lab Manual - P211 Wiki}
			\bibitem{ceres}
				\small{Ceres Solver, an Open-Source C++ optimization library\\
				http://ceres-solver.org}

			\setbeamertemplate{bibliography item}[book]
			\bibitem{taylor}
				\small{An Introduction to Error Analysis - John Taylor}

		\end{thebibliography}
	}

\end{document}
