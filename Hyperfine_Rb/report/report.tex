\documentclass{article}
\usepackage[margin=.5in]{geometry}
\usepackage{graphicx, dblfloatfix}
\usepackage{amsmath, amssymb, amsfonts, mathrsfs, mathtools}
\usepackage[english]{babel}
\usepackage[autostyle, english = american]{csquotes}
\usepackage[normalem]{ulem}
\usepackage[title,titletoc,toc]{appendix}
\usepackage{pgfplotstable}
\usepackage{array, booktabs, colortbl}
\MakeOuterQuote{"}

\pgfplotsset{compat=1.12}


\newcommand{\redchi}{$\tilde{\chi}^2\,$}
\newcommand{\twohalfmf}[2]{$^2 #1_{1/2},\, f = 2,\, m_f = #2$}
\newcommand{\twohalf}[1]{$^2 #1_{1/2},\, f = 2$}
\DeclareMathOperator{\erf}{erf}
\DeclareMathOperator{\cov}{cov}
\DeclarePairedDelimiter\abs{\lvert}{\rvert}%
\DeclarePairedDelimiter{\parens}{\lparen}{\rparen}

\title{Optical Pumping}
\author{Aman LaChapelle}

\begin{document}
\raggedright
\maketitle

\begin{abstract}
  In this experiment we will show that doppler broadening due to thermal fluctuations can be alleviated in order to show spectral features that would otherwise be hidden.  It is essential that everything is well aligned, but it is furthermore possible to 
\end{abstract}

\tableofcontents
\newpage

\section{Introduction}
