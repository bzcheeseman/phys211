\documentclass{article}
\usepackage[margin=.5in]{geometry}
\usepackage{graphicx, dblfloatfix}
\usepackage{amsmath, amssymb, amsfonts, mathrsfs, mathtools, physics}
\usepackage[english]{babel}
\usepackage[autostyle, english = american]{csquotes}
\usepackage[normalem]{ulem}
\usepackage[title,titletoc,toc]{appendix}
\usepackage{pgfplotstable}
\usepackage{array, booktabs, colortbl}
\MakeOuterQuote{"}

\pgfplotsset{compat=1.12}


\newcommand{\redchi}{$\tilde{\chi}^2\,$}
\DeclareMathOperator{\cov}{cov}
\DeclarePairedDelimiter{\parens}{\lparen}{\rparen}

\title{Neutron Studies}
\author{Aman LaChapelle}

\begin{document}
\raggedright
\maketitle

\begin{abstract}
  Here we show a method by which we can measure an effective radius for the nucleus of several different atomic masses, as well as observe the reaction that creates a deuteron.  We demonstrate techniques which can be used in order to perform these measurements using fast neutrons emitted from a PuBe source.
\end{abstract}

\tableofcontents
\newpage

\section{Introduction}
  In this experiment we demonstrate means by which we are able to observe the reaction that creates a deuteron, as well as measuring the size of the nucleus for several different atoms.  Specifically, we are using Cu, Al, C, and Pb absorbers.  We collect data for multiple thicknesses of each absorber in order to determine the scattering cross-section of the fast neutrons as they pass through the absorbers.  Based on the intensity of neutrons that hit the plastic scintillator detector, we are able to determine how many were absorbed or back-scattered.  From this intensity, we are able to calculate the size of the nucleus since neutrons will scatter only if they pass within about one wavelength (of the neutron's de Broglie wavelength) of the nucleus.  Thus, based on the proportion of neutrons that pass through the absorber we will be able to determine an approximate density (and thus size) for the nuclei of that absorber.

\section{Theory}
  \subsection{Deuteron Production}
  

  \subsection{Neutron Cross-Section/Nucleus Size}

\section{Experimental Procedure}

\section{Data and Uncertainty Analysis}

\section{Conclusion}

\end{document}
