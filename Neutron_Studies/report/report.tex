\documentclass{article}
\usepackage[margin=.5in]{geometry}
\usepackage{graphicx, dblfloatfix}
\usepackage{amsmath, amssymb, amsfonts, mathrsfs, mathtools, physics}
\usepackage[english]{babel}
\usepackage[autostyle, english = american]{csquotes}
\usepackage[normalem]{ulem}
\usepackage[title,titletoc,toc]{appendix}
\usepackage{pgfplotstable}
\usepackage{array, booktabs, colortbl}
\MakeOuterQuote{"}

\pgfplotsset{compat=1.12}


\newcommand{\redchi}{$\tilde{\chi}^2\,$}
\DeclareMathOperator{\cov}{cov}
\DeclarePairedDelimiter{\parens}{\lparen}{\rparen}

\title{Neutron Studies}
\author{Aman LaChapelle}

\begin{document}
\raggedright
\maketitle

\begin{abstract}
  Here we show a method by which we can measure an effective radius for the neutron, as well as observe the reaction that creates a deuteron.  We demonstrate techniques which can be used in order to perform these measurements using fast neutrons emitted from a PuBe source.
\end{abstract}

\tableofcontents
\newpage

\section{Introduction}
  In this experiment we demonstrate means by which we are able to observe the reaction that creates a deuteron, as well as measuring the size of the neutron.  We collect data for multiple thicknesses of each absorber in order to determine the scattering cross-section of the fast neutrons as they pass through the absorbers.  Based on the intensity of neutrons that hit the plastic scintillator detector, we are able to determine how many were absorbed or back-scattered.  From this intensity, we are able to calculate the size of the nucleus since neutrons will scatter only if they pass within about one wavelength (the neutron's de Broglie wavelength) of the nucleus.  Thus, based on the proportion of neutrons that pass through the absorber we will be able to determine an approximate density (and thus size) for the nuclei of that absorber.

\section{Theory}
  \subsection{Deuteron Production}
  There is a stable bound state of a proton and a neutron called a deuteron.  In general, if the neutron has energy comparable to the proton, it will be possible to have the neutron hit the proton and create this bound state.  It differs from deuterium in that it does not necessarily have the bound electron that a hydrogen atom typically does.  The important concept to note is that the deuteron is a bound state of a neutron and of a proton.  This state has lower energy than the sum of the incoming energy of the neutron and proton, and so the reaction that creates a deuteron is characterized by a capture photon on the order of 2 MeV.  Since the neutrons and protons that have low enough energy to fall into a bound state like this one are generally barely moving if at all, we take their energy to be roughly equal to their rest mass for the purposes of a rough calculation.  Use the following reaction:
  \begin{equation*}
    n + p \rightarrow d + \gamma
  \end{equation*}
  We determine that given the rest energy of the neutron, proton, and deuteron, the capture photon should have an energy of around 2.1 MeV.

  \hspace{.25cm}

  Experimentally, this reaction is not particularly simple to measure.  However, we take a hydrogen-rich substance - paraffin - and place it in the path of the fast neutrons.  We then place that in front of our detector which we expect to measure the capture photons.  However, there is a lot of ambient energy, which translates into a lot of ambient photons.  This translates into, experimentally, shielding the detector with lead bricks in order to attenuate photons coming from the paraffin surrounding the PuBe core, or even ambient lighting, for example.  Thus we take a series of spectra to determine background rates, and we will perform a check to make sure that the deuteron production is actually occurring with the hydrogen atoms in the paraffin and not (for example) the creation and decay of Carbon-13.

  \subsection{Neutron Cross-Section/Nucleus Size}
  

\section{Experimental Procedure}

\section{Data and Uncertainty Analysis}

\section{Conclusion}

\end{document}
