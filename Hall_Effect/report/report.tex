\documentclass[reprint, nobibnotes, amssymb, amsmath, amsfonts, physics, mathtools, mathrsfs, floatfix]{revtex4-1}
\usepackage{graphicx}
\usepackage[english]{babel}

\newcommand{\redchi}{$\tilde{\chi}^2\,$}

\begin{document}

  \title{Relativistic Dispersion of Free Electrons}

  \author{Aman LaChapelle}
  \affiliation{The University of Chicago}

  \begin{abstract}
    We demonstrate the Hall Effect in a Germanium sample that is p-type doped with Gallium.  We observe hole conduction at low temperatures and measure a Hall coefficient of XXXXXXXX that has XXXXXXX correlation with temperature.  We measure an energy gap between valence and conduction bands in $Ge_{1-x}Ga_x$ of XXXX.
  \end{abstract}

  \maketitle
  \tableofcontents

  \section{Introduction and Theory}
    The Hall effect is an effect that occurs when we run a current through a region with a perpendicular magnetic field.  The classical picture is that the electrons which move in cyclotron orbits in the bulk will move in a skipping motion at the edge and create edge currents, significantly, a current that induces a voltage difference that is perpendicular to the applied current/voltage.  This effect was famously first noticed by Edwin Hall in 1879~\cite{classical_hall}.  Later on, after the formalism proposed by Landau and Lifschitz ~\cite{landau}, and the nearly concurrent discovery by Von Klitzing of the exact quantization of the Hall conductivity~\cite{Von_Klitzing} did the solid state physics community begin to understand the quantum effects behind the Hall effect.

    Shortly after the discovery of the Integer Quantum Hall Effect (IQHE), came the discovery of what is now called the Fractional Quantum Hall Effect (FQHE).  The first observation of the FQHE was by Horst St\"{o}rmer and Dan Tsui~\cite{stormer_tsui}, searching for the theorized Wigner crystal that would form from electrons in a solid under extremely high mangetic fields.  Just a year later Robert Laughlin introduced the Laughlin wave function that did an excellent job of explaining this effect - now understood to be because of fractional filling of the landau levels.

    Subsequent discoveries in the realm of quantum hall physics have mostly focused on topological nontriviality, and a current prevailing explanation uses Chern-Simons field theory to perform a singular gauge transformation in order to attach flux quanta to electrons in the form of vortices - a.k.a. zeros of the wave function.  We can think of the effect in a very simplistic way as electrons and vortices bound together in such a way that the electron wave function picks up a phase of $2\pi$ times the number of vortices that the electron is pinned to as it goes around.  This is (again very simplistically) an extension of the Aharonov-Bohm effect, and is beginning to be understood in terms of topological nontrivialities that are properties of the quantum system itself - another interpretation being the relation of the Berry phase to the Chern number of the system and characterizing its topological invariance.  It is important to recognize that much of this characterization occurs in real-space, which allows real-space measurements of system invariants such as the Chern number and other topological invariants.

    For our case, the classical theory provides enough formalism, as we are not dealing with a temperature, experimental resolution, or field at which partial filling factors come into play, we are only concerned with the voltage difference induced perpendicular to the induced voltage at a very classical level.  The essence of the Hall effect remains much the same - we will notice that the cause of the hall voltage is still due to the landau quantization (however at high filling factors/currents it is nearly continuous) and the edge states that arise as a result of this quantization.  We can, however, describe much of this formalism in the classical limit with Dr\"{u}de theory as follows.

    If we consider electrons/holes as simple masses traveling through a region with a certain net velocity (drift velocity) and time between scattering events (mean free time), where the current is described by
    \begin{equation}
      j = -nev_d
    \end{equation}
    we can then define a conductivity tensor and simply write down the Lorentz force equation to return the Hall Voltage and Resistance as the following:
    \begin{gather}
      R_H = \frac{1}{q n} = \frac{E_y}{j_x B_z} \label{hall_resist} \\
      V_H = R_H j_x B_z w \label{hall_volt}
    \end{gather}
    with $j_x$ the applied current and w the width of the sample.  It is important to note that we have defined the Hall Resistance in terms of the charge, $q$, when it is normally defined in terms of the electron charge, $-e$, or the hole charge, $+e$, depending on the primary charge carrier.

    Now, with a basic understanding of band theory, we can rewrite this a little in order to take better accounting of electrons and holes and potentially different scattering characteristics.  We rewrite the current (and the drift velocity) in terms of the mobility and get the following equations that allow us to take into account different charge carriers in different systems:
    \begin{gather}
      \mu_{h/e} = \frac{e \tau_e/h}{m_{e/h}^*} \label{mobility} \\
      \text{Which leads us to the following:} \nonumber \\
      \sigma = e(n_h\mu_h + n_e\mu_e) \label{conductivity} \\
      R_H = \frac{E_y}{j_x B_z} = \frac{\mu_h^2 n_h - \mu_e^2 n_e}{\sigma^2/e}
    \end{gather}
    with $\sigma$ being the conductivity we are familiar with from Ohm's law.  This formulation suggests the definition of a new quantity, the Hall mobility:
    \begin{equation}
      \mu_H = |R_H|\sigma \label{hall_mobility}
    \end{equation}
    which allows us to relate resistivity of a given semiconductor in and out of a magnetic field in a phenomenon known as magnetoresistance - which is the same phenomenon wherein we see peaks of resistivity at the quantized hall plateaus - the resistance in the induced direction increases sharply as we come to quantized values for the Hall resistance.

    p-type semiconductors

    \section{Experimental Methods}
    stuff goes here



  \bibliography{bibliography}
  \bibliographystyle{plain}

\end{document}
